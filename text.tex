\lec{1}{February 11, Thursday}

\section{Note}

For the rest of this course, $\N$ contains zero.
For any set $S$, we think that $S^0 = \b{0} = \b{\naught}$.

\section{Partial recursive functions}

\begin{definition}
  \label{simplistic}
  Let $f \col {\ss} \N^k \to \N$ be a partial function.
  $f$ is \emph{simplistic}, iff
  \begin{enumerate}
    \item $f(x) = 0$ (\emph{zero}, $f =: 0$).
    \item $f(x) = x+1$ (\emph{successor}, $f =: s$).
    \item $f(x_1, \dots, x_n) = x_m$ (\emph{projection}, $f =: I^n_m$)
  \end{enumerate}
\end{definition}
  
\begin{definition}
  There are several operations with functions $\N^k \to \N$, each of which we assign a letter. 
  
  The \emph{composition operator} $S$. If we have $h(y_1, \dots, y_m)$ and $g_i(x_1, \dots, x_n)$, $i=1, \dots, n$, we define their \emph{composition}
  $f$ as
  $$ f = h\p{g_1(x_1, \dots, x_n), \dots, g_m(x_1, \dots, x_m)}. $$
  
  The \emph{primitive recursion} operator $R$.
  $f$ of arity $n+1$ is defined with $g$ and $h$ of arities $n$ and $n+2$ as
  \begin{align*}
    f(x_1, \dots, x_n, 0) &= g(x_1, \dots, x_n) \\
    f(x_1, \dots, x_n, y+1) &= h\p{x_1, \dots, x_n, y, f(x_1, \dots, x_n, y)}.
  \end{align*}
  
  $f$ is said to be a \emph{primitive recursive} function, iff there exists $f_1, \dots, f_k$ --- a sequence of functions such that $f_i$ is either simplistic, or gotten from $f_1, \dots, f_{i-1}$ with the help of $S$ and $R$; and $f_k = f$.
\end{definition}

\begin{example}
  $f(x, y) = x+1$ is primitive recursive:
  $$
  \begin{cases}
    f(x, 0) = x = I^1_1(x), \\
    f(x, y+1) = (x+y)+1 = s\p{f\p{x, y}} = s\p{I_3^3\p{x, y, f(x, y)}},
  \end{cases}
  $$
  so we can put $g = I^1_1$ and $h = s \circ I^3_3$ in the definition above. 
\end{example}

\begin{lemma}
  The following are primitive recursive:
  \begin{enumerate}
    \item Constants.
    \item Binary sums, products, powers.
    \item $[x \ne 0]$.
    \item $[x = 0]$.
    \item $ (x-1)[x > 0].$
    \item $ (x-y)[x \ge y]$.
    \item $ \a{x-y}$.
  \end{enumerate}
\end{lemma}

\begin{proof}
  ~\begin{enumerate}
    \item Suppose $f \col A \to \N$, where $A \ss \N^k$, is $c \in \N$ everywhere. If $c = 0$, then $f$ is simplistic, and is primitive as such. Suppose $c > 0$. By induction, the function $g \col A \to \N^k$ that maps $x \mapsto c-1$ is primitive. We then have
    $$ f(x) = s(g(x)) $$
    for any $x \in A$, so $f$ is primitive by the composition rule.
    \item For sums this has been shown in the preceding example. Let $f(x, y) = xy$.
    \begin{align*}
       f(x, 0) &= 0, \\
       f(x, y+1) &= x(y+1) = xy+x = f(x, y)+x.
    \end{align*}
    Since sums are primitive, $f$ is primitive by the recursion rule.
    \item Let $f(x, y) = x^y$.
    \begin{align*}
      f(x, 0) &= 1, \\
      f(x, y+1) &= f(x, y) \cdot y.
    \end{align*}
    Since products are primitive, $f$ is primitive by the recursion rule.
    \item Let $f(x) = [x \ne 0]$. Let $h \col A \to \N$ be the constant 1. Then
    \begin{align*}
      f(0) &= 0, \\
      f(y+1) &= 1 = h(y).
    \end{align*}
    Recursion.
    \item Let $f(x) = [x = 0]$. Then
      $$ f(x) = 1-[x\ne 0]. $$
      The function $g(x) = 1-x$, defined on $\b{0,1}$, is primitive by a trivial application of the recursion rule. Hence $f$ is, by the composition rule.
    \item Let $f(x) = (x-1)[x > 0]$. We denote $f(x) = x \dot{-} 1$.
    \begin{align*}
      f(0) &= 0, \\
      f(x+1) &= x.
    \end{align*}
    $f$ is primitive by the recursion rule (the identity function is the projection $I^1_1$).
    \item Let $f(x, y) = (x-y)[x \ge y] = x \dot - y.$
    Observe that
    $$ x \dot - (y+1) = (x \dot - y) \dot - 1, $$
    so
    \begin{align*}
      f(x, 1) &= (x-1)[x \ge 1] = (x-1)[x > 0], \\
      f(x, y+1) &= \p{f(x,y)-1}\q{f(x, y) > 0}. 
    \end{align*}
    \item Let $f(x, y) = \a{x-y}$. Then
    \begin{align*}
      f(x, y)
      & = \max(0, x-y) - \min(0, x-y) \\
      & = \max(0, x-y) + \max(0, y-x) \\
      & = \p{x \dot - y } + \p{y \dot - x}.
    \end{align*}
  \end{enumerate}
\end{proof}

\subsection{Minimisation and partial recursive functions}

\begin{definition}[minimisation operator $\mu$]
  If $g$ is a function of arity $n+1$, we may construct $f$ of arity $n$ as
  \begin{align*}
    f\p{x_1, \dots, x_n} = \min \b{ y \mid g\p{x_1, \dots, x_n, y} = 0 }.
  \end{align*}
\end{definition}

\begin{example}
  $$
  x \dot - y = \min \b{ z \mid \a{(x-y) -z} = 0 }.
  $$
\end{example}

%\begin{definition}
%  $f$ is \emph{partial recursive}, iff there is a sequence $f_1, \dots, f_k$ such that $f_i$ are either simplistic, or gotten from $f_1, \dots, f_{i-1}$ by applying one of the three operators.
%\end{definition}
%
%\begin{remark}
%  Primitive recursive functions are defined everywhere, partial recursive ones --- not necessarily.
%  Also, there exists a function that is defined everywhere, but is not primitive.  
%\end{remark}

\subsection{Bounded minimisation}

\begin{notation}
    $\ol x = \p{x_1, \dots, x_n}.$
\end{notation}

\begin{definition}[bounded minimisation operator $\mu_{\le}$]
  If $g$ and $h$ are functions of arity $n+1$ and $n$, respectively, we may construct partial $f$ as
  $$ f(\ol x) = \min \b{ y \mid g(\ol x, y) = 0,\ y \le h(\ol x)}. $$
\end{definition}

\begin{lemma}
  If $f \in \PR$, binary operation $\odot \in \PR_{n+1}$ is associative, and
  $$ g(\ol x, y) = \bigodot_{i=0}^y f(\ol x, i), $$
  then $g \in \PR$.
\end{lemma}

\begin{proof}
  \begin{align*}
    g(\ol x, 0) &= f(\ol x, 0), \\
    g(\ol x, y+1) &= g(\ol x, y) \odot f(\ol x, y).
  \end{align*}
\end{proof}

\begin{statement}
  If $g$ and $h$ are total and primitive recursive, and $f$ is as in the previous definition, then $f$ is primitive recursive.
\end{statement}

\begin{proof}
  $$ f(\ol x) = \sum_{i=0}^{h(\ol x)} \prod_{j=0}^i \q{ g(\ol x, j) \ne 0 }. $$
\end{proof}

\section{Primitive recursive predicates}

\begin{definition}
  The predicate $T$ is called \emph{primitive recursive}, iff its characteristic function $x \mapsto [T(x)]$ is primitive recursive. 
\end{definition}

\begin{proposition}
  If $P$ and $Q$ are primitive recursive predicates, then
  $\neg P$,
  $P \lor Q$, $P \land Q$, $P \Rightarrow Q$ are primitive recursive. 
\end{proposition}

\begin{proof}
  The last statement is superfluous, but we write the formula, nevertheless:
  \begin{align*}
    [\neg P] &= 1-[P], \\
    [P \land Q] &= [P] [Q], \\
    [P \lor Q] &= [P] + [Q] - [P][Q], \\
    [P \Rightarrow Q] &= [\neg P \lor Q] \\
    &= 1-[P][\neg Q].
  \end{align*}
\end{proof}

\begin{proposition}
  $=$, $\le$, $\ge$, $<$, $>$ are primitive recursive predicates.
\end{proposition}

\begin{proof}
  ~\begin{enumerate}
    \item $[x = y] = \q{\a{x-y} = 0}.$
    \item Let $f(x, y) = [x \le y]$. Then
    \begin{align*}
      f(x, 0) &= 0, \\
      f(x, y+1) &= [x \le y+1] \\
      &= [x \le y] + [x = y+1] \\
      &= f(x, y) + [x = y+1].
    \end{align*}
    Now recall the point 1.
    \item Composing with projections, we swap arguments of $\le$ to get $\ge$.
    \item $[x < y] = [x \le y] \cdot [x \ne y]$.
    \item $[x > y] \in \PR$ by the same token as with $\ge$.  
  \end{enumerate}
\end{proof}

\begin{lemma}
  Let $R \ss \N^{n+1}$ be a primitive recursive predicate.
  Then the predicates
  \begin{align*}
    \ex i \le y \col & R\p{\ol x, i}, \\
    \fa i \le y \col & R\p{\ol x, i}, \\
    \ex i < y \col & R\p{\ol x, i}, \\
    \fa i < y \col & R\p{\ol x, i} \\
  \end{align*}
  are primitive recursive.
\end{lemma}

\begin{proof}
  For the first one, observe that
  \begin{align*}
    \q{\ex i \le y \col R\p{\ol x, i}}
    &= \bigvee_{i=0}^y \q{R\p{\ol x, i}}. 
  \end{align*}
  $\vee$ is an associative operation.
  
  Likewise,
  \begin{align*}
    \q{\fa i \le y \col R\p{\ol x, i}}
    &= \bigwedge_{i=0}^y \q{R\p{\ol x, i}}.
  \end{align*}
  
  The last two predicates are gotten by composing the first two ones with $y \dm 1$.
\end{proof}


\section{Applications of minimisation}

\begin{lemma}
  The functions
  \begin{enumerate}
    \item $\f{\frac{x}{y}}$,
    \item $[x \shortmid y]$,
    \item $[x \in \P]$,
    \item $p_x = \p{\text{the prime $x$ in order}}$
  \end{enumerate}
  are primitive recursive.
\end{lemma}

\begin{proof}
  ~\begin{enumerate}
    \item $\f{x/y} = \min \b{ q \mid x < (q+1)y,\ q \le xy }$ (we need the second condition for the minimisation to be bounded). Since multiplication and comparisons are primitive, the predicate is primitive.
    \item $[x \smid y] = \q{x=y\cdot\f{x/y}}.$
    \item Let $f(x)$ be the minimal divisor of $x$ that differs from 1. Then $[x \in \P] = \q{f(x) = x}$, and
    \begin{align*}
      f(x)
      &= \min \b{ d \mid d \smid x, \ d \ne 1, \ d \le x }.
    \end{align*}
    \item The equations
    \begin{align*}
      p_0 &= 2, \\
      p_{x+1} &= \sum_{i=0}^{p_x!+1} \prod_{j=0}^i \q{j \not \in \P \lor j \le p_x}
    \end{align*}
    define $p_\square$, since there is at least one prime in the sum, $p_x!+1 \in \P$.
  \end{enumerate}
\end{proof}

\begin{lemma}
  The function
  $$ x \mapsto \text{undefined} $$
  is primitive.
\end{lemma}

\section{Mutual and complete recursion}

\begin{lemma}
  The function $x \choose 2$ is primitive.
\end{lemma}

\begin{proof}
  Indeed,
  \begin{align*}
    {0 \choose 2} &= 0, \\
    {x+1 \choose 2} &= {x \choose 2} + x.
  \end{align*}
\end{proof}

\begin{definition}
  Call $f \col \N^n \to \N$ a \emph{Cantor enumeration}, iff it is bijective, primitive recursive, and has all coordinate functions of the inverse primitive recursive.
\end{definition}

\begin{lemma}
  Define the \emph{} $f \col \N^2 \to \N$ as
  $$ f(x, y) = {x+y+1 \choose 2}+y. $$
  Then $f$ is a Cantor enumeration.
\end{lemma}

An irrelevant note: ${x+y+1 \choose 2}$ is the number of cells before the diagonal number $x+y$ from the origin. $y$ is the height of the cell $(x, y)$ on this diagonal.

\begin{proof}
  ${x+y+1 \choose 2}$ is the greatest triangular number not surpassing $f(x, y)$: if the next one, ${x+y+2 \choose 2}$, is $\le f(x, y)$ (they are monotonous, since there is an injection of pairs), then
  $$ \sum_{i=0}^{x+y+1} i \le y + \sum_{i=0}^{x+y}i \iff x+y+1 \le y \iff x+1 \le 0, $$
  which is hardly true for natural $x$. Hence $x+y$ is uniquely determined, as is $y$. This we use to write down the inverses $g_x, g_y$. Put
  \begin{align*}
    g_s(z) &= \min \b{t\ \middle|\ {t+2 \choose 2} > z,\ t \le z}, \\
    g_y(z) &= z - {g_s(z) \choose 2}, \\
    g_x(z) &= g_s(z) - g_y(z).
  \end{align*}
  ${z+2 \choose 2} > z$, since each of the $z$ initial elements gives rise to a pair with the element number $z+1$, and there is a pair which consists of the last two elements. Therefore, $g_s$ is defined everywhere.
\end{proof}

\begin{theorem}
  For each $n \in \N_{\ge 1}$ there exists a Cantor enumeration of $\N^n$.  
\end{theorem}

(For $n = 0$, $\N^n$ is finite.)

\begin{proof}
  By induction over $n$. In case $n = 1$, we have $\Id_\N$. Let $f$ and $g$ be Cantor enumerations of $\N^n$ and $\N^2$. Define an enumeration $h$ of $\N^{n+1}$ as
  \begin{align*}
    h\p{x_1, \dots, x_{n+1}} &= g\p{f\p{x_1, \dots, x_n}, x_{n+1}}.
  \end{align*}
  Since $f_i^{-1}$ are functional and primitive by induction, we see that
  \begin{align*}
    x_i &= f^{-1}_i\p{g_1^{-1}\p{h}} \text{ for all } i \in \b{1,\dots,  n}, \\
    x_{n+1} &= g_2^{-1}\p{h}.
  \end{align*}
\end{proof}

\begin{definition}
  Denote
  $$ \Ex\p{i, x} = \max \b{k\ \middle|\ p_i^k \smid x}. $$
\end{definition}

\begin{lemma}
  $\Ex \in \PR_2$.
\end{lemma}

\begin{proof}
  Since
  $$ \Ex\p{i, x} = \min\b{k\ \middle|\ p_i^{k+1} \not\shortmid x,\ k \le x}. $$
  $p_i^{x+1} \snmid x$, since $b^x > x$ for $b \ge 2$.
\end{proof}

\begin{theorem}[complete recursion]
  Let $s \in \N_{\ge 1}$, $g \in \PR_n$, $h \in \PR_{n+2}$, $t_1, \dots, t_s \in \PR_1$, $t_i(y) \le y$ for all $i \in \b{1, \dots,  s}$.
  Define $f$ as
  \begin{align*}
    f\p{\ol x, 0} &= g\p{\ol x}, \\
    f\p{\ol x, y+1} &= h\p{\ol x, y, f\p{\ol x, t_1(y)}, \dots, f\p{\ol x, t_s(y)}}.
  \end{align*}
  Then $f \in \PR_{n+1}$.
\end{theorem}

\begin{proof}
  To simplify notation, we put $n = 0$ (the proof would be the same anyway).
  Using primitive recursion, define
  \begin{align*}
    q(0) &= 2^{g}, \\
    q(x+1) &= q(x) \cdot p_{x+1}^{h\p{x, \Ex\p{t_1(x), q(x)}, \dots, \Ex\p{t_s(x), q(x)}}}.
  \end{align*}
  Obviously,
  $$ f(x) = \Ex\p{x, q(x)} $$
  for all $x \in \N$ --- a primitive function.
\end{proof}

\begin{theorem}[mutual recursion]
  For $i \in \b{1, \dots, k}$ and some $g_1, \dots, g_k, h_1, \dots, h_k \col \N^n \to \N$, define
  \begin{align*}
    f_i(\ol x , 0) &= g_i(\ol x), \\
    f_i(\ol x, y+1) &= h_i\p{\ol x, y, f_1\p{\ol x, y}, \dots, f_k\p{\ol x, y}}.
  \end{align*}
  Suppose $g_i, h_i$ for $i \in [1, s]$ are primitive recursive. Then $f$ is primitive recursive.
\end{theorem}

\begin{proof}
  Let $c \col \N^n \to \N$ be a Cantor enumeration, and, for every $i \in \b{1, \dots, n}$, $p_i \col \N \to \N$ its $i$th inverse. Define
  $$ f\p{\ol x, y} = c\p{f_1\p{\ol x, y}, \dots, f_n\p{\ol x, y}}. $$
  We assert $f \in \PR_{n+1}$: if this is settled, $f_i = p_i \circ f$ are primitive as well.
  First, define 
  \begin{align*}
    \wh h_i\p{\ol x, y, z} &:= h_i\p{\ol x, y, p_1(z), \dots, p_n(z)} \text{ for any $i \in \b{1, \dots, n}$}, \\
    h\p{\ol x, y, z} &:= c\p{\wh h_1\p{\ol x, y, z}, \dots, \wh h_n\p{\ol x, y, z}}.
  \end{align*}
  This $h$ is a primitive function.
  And now we have made our way to applying the recursion rule:
  \begin{align*}
    f\p{\ol x, 0} &= c\p{g_1\p{\ol x}, \dots, g_n\p{\ol x}}, \\
    f\p{\ol x, y+1} &= h\p{\ol x, y, f\p{\ol x, y}}.
  \end{align*}
\end{proof}

\begin{theorem}
  Let $R_0, \dots, R_k$ be $n$-ary relations such that
  $$ R_0 \sqcup \dots \sqcup R_k = \N^n. $$
  For some $f_1, \dots, f_k \col \N^n \to \N$, define
  $$
  f(\ol x) =
  \begin{cases}
    f_0(\ol x), & R_0(\ol x), \\
    \vdots \\
    f_k(\ol x ), & R_k(\ol x).
  \end{cases}
  $$
  Suppose $f_i$ and $R_i$ are primitive recursive. Then $f$ is primitive recursive.
\end{theorem}

\begin{proof}
  Indeed,
  $$ f(\ol x) = \sum_{i=0}^k f_i(\ol x) \q{R_k(\ol x)}. $$
\end{proof}

\section{Computable functions}

\newcommand{\Comp}[0]{\on{R}}

\begin{definition}
  Let $D \ss \N^m$.
  A function $f \col D \to \N^n$ is \emph{computable}, iff there exists a TM that, starting with any $x \in D$ written on its input tape, stops with only $f(x)$ written on the output tape.
  We denote by $\Comp_m$ the set of all computable partial functions ${\ss\N^m} \to \N$, and by $\Comp_m^* \ss \Comp_m$ the set of computable functions $\N^m \to \N$.
\end{definition}

\begin{definition}
  A set $X \ss \N^k$ is \emph{decidable}, iff its characteristic function is computable.
\end{definition}

\begin{lemma}
  Let $f \col \N \to \N$ be a constant almost everywhere function. Then $f$ is computable.
\end{lemma}

\begin{proof}
  Indeed, it is a primitive recursive function: if $f|_{[t, +\infty)}$ is constant, then
  $$ f(x) = f(x) \q{x \ge t} + \sum_{i=0}^{t-1} f(i) \q{x=i}. $$
\end{proof}

\begin{example}
  Let $S \ss \N$ be the set of such $n$ that the decimal expansion of $e$ contains $n$ consecutive nines.
  Then $S$ is decidable, since its characteristic function is nondecreasing. 
\end{example}

\begin{lemma}
  An infinite set $A \ss \N$ is decidable iff there exists a computable increasing function $f \col \N \to \N$ such that $A = \im f$.
\end{lemma}

\begin{proof}
  Suppose $A$ is decidable. Define $f$ as 
  $$
  f(x)=
  \begin{cases}
    \min \b{a \mid a \in A,\ a > f(x-1)}, & x > 0, \\
    \min \b{a \mid a \in A}, & x = 0.
  \end{cases}
  $$
  By what we know about minimisation, this function is indeed computable.
  By its definition, it is increasing.
  Its image is the complete $A$: otherwise take the smallest natural $n \in A \sm \im f$; all the lesser elements of $A$ are in the image; then there exists $x$ such that $f(x-1)$ is the largest number in the image; but then $f(x) = n$. Finally, $f$ is defined everywhere, since otherwise $A$ would be finite.
  
  Conversely, suppose that there exists such a function.
  To compute $\q{x \in A}$ for any $x \in \N$, check all values of $f$ until we reach one that is at least this $x$. The definition of $f$ allows us to do that for all $x$.
\end{proof}

\section{Equivalence of Kleene and Turing computability}

\begin{theorem}
  $f$ is computable iff it is partial recursive.
\end{theorem}

\begin{proof}[Proof of $\sps$]
  Suppose the function $f$ is partial recursive.
  We agree to represent a tuple of arguments $\ol x$ to $f$ as
  $$ 0 \prod_{i=1}^n 1^{x_i} 0. $$
  The proof is as follows:
  \begin{enumerate}
    \item \emph{There is a machine for each of the simplistic functions.} Easy to see.
    \item \emph{There is a machine for composition of functions.} In terms of the composition operator,
    copy the input $n$ times; 
  run TMs for the functions $g_1, \dots, g_n$;
  run the TM for $h$ on the result.
    \item \emph{There is a machine for functions which are constructed by primitive recursion.}
    \begin{align*}
      M_1 \col & (\ol x, y) \mapsto \p{\ol x, g\p{\ol x}}, \\
      M_2 \col & \p{y, \ol x, u, z} \mapsto \p{y, \ol x, u+1, h\p{\ol x, u, z}}, \\
      M_3 \col & \p{y, \ol x, u, z} \mapsto \p{z}, \\
      M_4 \col & \p{y, \ol x, u, z} \mapsto \p{\q{u \ne y}}.
    \end{align*}
    Now the wanted machine can be built as
    $$ \text{$M_1$; while $M_4$ do $M_2$; $M_3$}. $$
    \item \emph{There is machine for functions constructed by bounded minimisation.}
    Let
    \begin{align*}
      N_1 \col & \p{\ol x} \mapsto \p{\ol x, 0}, \\
      N_2 \col & w \mapsto w \# w, \\
      N_3 \col & \p{\ol x, y} \# \p{\ol x, y} \mapsto \p{\ol x, y} \# \p{g\p{\ol x, y}}, \\
      N_4 \col & w \# v \mapsto \q{v \ne \eps}, \\
      N_5 \col & \p{\ol x, y} \# w \mapsto \p{y}, \\
      N_6 \col & \p{\ol x, y} \# w \mapsto \p{\ol x, y+1}.
    \end{align*}
    The sought for machine is
    $$ \text{$N_1$, $N_2$, $N_3$; while $N_4$ do $N_6, N_2, N_3$; $N_5$}. $$
  \end{enumerate}
\end{proof}

\begin{proof}[Proof of $\ss$]
  Suppose we have $m$ symbols in the alphabet $\Gamma = \b{a_0, \dots, a_{m-1}}$.
  We code configurations as
  $$ \ga q a \gb \mapsto \p{\wh \ga, q, \wh a, \wt \gb}, $$
  where $\wh \square$ is the number in base $m$ which is written as $\s$; $\wt \s = \wh{\s^R}$.
   
   By a pair $(q, a) \in Q \times \Gamma$ we can determine the action of the machine, and this will be a PR function (since it takes meaningful values on only a finite set).
   
   We can transform a configuration by a PR function. For example, if the head moved right, the number $\wh \ga$ becomes $m \cdot \wh \ga + \wh a$. The new symbol $\wh a$ is found, in this case, by computing $\wh \gb \mathrel{\%} m$, and the new string $\wh \gb$ as $\f{\wh \gb / m}$.
   
   We can encode the work of the complete machine using mutual recursion. Define the functions $K$, $K_\ga$, $K_\gb$, $K_a$, $K_q$ that compute the elements of the next configuration, based on the previous one. The last parameter of each is some $t$, so we compute their values on $t+1$, referring to the ones on $t$.
   
   We can now find the first moment $t_f$, on which a final state is reached, by using minimisation on $K_q$. Afterwards we compute $K_a\p*{t_f}$ and $K_b(t_f)$ to find the computation result (wlog, the machine stops with $\ga = \eps$).
\end{proof}

\begin{corollary}
  Any partial recursive function can be computed using at most one minimisation.
\end{corollary}

\begin{corollary}
  A function, which is computable on a Turing machine in time $O(f)$ where $f$ is primitive recursive, is primitive recursive.
\end{corollary}

\section{The Ackermann function}

In this section, all powers are functional powers.

\begin{definition}
    Define
  \begin{align*}
    \ga_0(x) &= x+1, \\
    \ga_i(x) &= \ga_{i-1}^{n+2}(x).
  \end{align*}
  The \emph{Ackermann function $\gb \col \N \to \N$} is then defined as
  \begin{align*}
    \gb(x) &= \ga_x(x).
  \end{align*}
\end{definition}

We assert that the function $\gb$ grows faster than any primitive recursive functions. Yet it is computable (so partial recursive).

\begin{lemma}
  $\ga_i(x) > x$ for all $i, x \in \N$.
\end{lemma}

\begin{proof}
  For $i = 0$, $x+1 > x$. For $i> 0$,
  \begin{align*}
    \ga_i(x) &= \ga_{i-1}\p{\ga_{i-1}^{x+1}(x)} \\
    &> \ga_{i-1}^{x+1}(x) \\
    &\vdots \\
    &> x.
  \end{align*}
\end{proof}

\begin{lemma}
  If $x > y$, then $\ga_i(x) > \ga_i(y)$.
\end{lemma}

\begin{proof}
  By induction on $i$, then by induction on $x$.
  If $i = 0$,
  $$ x > y \implies x+1 > y+1. $$
  If $i > 0$, then
  \begin{align*}
    \ga_i(y)
    &= \ga_{i-1}\p{\ga_{i-1}^{n+1}(y)} \\
    &> \ga_{i-1}\p{\ga_{i-1}^{n+1}(x)} \\
    &= \ga_i(x).
  \end{align*}
\end{proof}

\begin{lemma}
  For every $x \in \N$, if $i > j$, then $\ga_i(x) > \ga_j(x)$.
\end{lemma}

\begin{proof}
  If $i = j+1$, then
  \begin{align*}
    \ga_{j+1}(x)
    &= \ga_j^{x+1}\p{\ga_j(x)} \\
    &> \ga_j(x),
  \end{align*}
  since $\ga_j(\s) > \s$.
\end{proof}

\begin{lemma}
  $\ga_i(x) > \ga_{i-1}\p{\ga_{i-1}(x)}$.
\end{lemma}

\begin{proof}
  \begin{align*}
    \ga_i(y)
    &= \ga_{i-1}^{n+1}\p{\ga_{i-1}(y)} \\
    &> \ga_{i-1}\p{\ga_{i-1}(x)}.
  \end{align*}
\end{proof}

\begin{lemma}
  Let $f \in \PR_n$. Then exists $k$ such that, for all $x_1, \dots, x_n \in \N$,
  $$ f(x_1, \dots, x_n) \le \ga_k\p{\max\p{x_1, \dots, x_n}}. $$
\end{lemma}

\begin{proof}
  By induction on the structure of primitive functions.
  
  Consider the simplistic functions.
  \begin{enumerate}
    \item If $f(x) = 0$, then $k = 0$ goes, since $x+1 > 0$.
    \item If $f(x) = x+1$, then $k = 0$ goes, since $\ga_1(x) \ge \ga_0(x) = f(x)$.
    \item If $f(\ol x) = x_i$, then $k = 0$ goes.
  \end{enumerate}
  
  Consider the composition operator.
  Suppose
  $$ f(\ol x) = h\p{g_1(\ol x), \dots, g_m(\ol x)}. $$
  By induction there exist $k$ and $l$ such that
  \begin{align*}
    h\p{g_1(\ol x), \dots, g_m(\ol x)}
    &\ge \ga_k\p{ g_i(\ol x)} \\
    &\ge \ga_k\p{\ga_l \p{g_i(\max \ol x)}} \\
    &> \ga_k\p{\ga_l \p{g_i(\max \ol x)}} \\
    &> \ga_k(\max \ol x).
  \end{align*}
  
  Consider the primitive recursion operator.
  Suppose 
  \begin{align*}
    f(x_1, \dots, x_n, 0) &= g(x_1, \dots, x_n) \\
    f(x_1, \dots, x_n, y+1) &= h\p{x_1, \dots, x_n, y, f(x_1, \dots, x_n, y)}.
  \end{align*}
  There exists $k$ such that
  \begin{align*}
    f(x_1, \dots, x_n, 0)
    &= g(x_1, \dots, x_n) \\
    &\le \ga_k\p{\max\b{x_1, \dots, x_n}}, \\
    \\ 
    f(x_1, \dots, x_n, y+1)
    &= h\p{x_1, \dots, x_n, y, f(x_1, \dots, x_n, y)} \\
    &\le \ga_k\p{\max\b{x_1, \dots, x_n, y, f(x_1, \dots, x_n, y)}} \\
    &\le \ga_k\p{\max\b{x_1, \dots, x_n, y+1}}
  \end{align*}
\end{proof}

\begin{theorem}
  Let $\gb(x) = \ga_x(x)$, $f \in \PR_n$. There exists $k \in \N$ such that $\gb(x) > f(x)$ for all $x > k$.
\end{theorem}

\begin{proof}
  By the previous lemma, there is $k$ such that
  $$ f(x) \le \ga_k\p{x}. $$
  If $x > k$, this inequality can be continued to yield
  $$ f(x) < \ga_x(x). $$
\end{proof}

\section{Some partial recursive functions}

\newcommand{\und}[0]{\text{undefined}}
\newcommand{\oth}[0]{\text{otherwise}}

\begin{lemma}
  The function
  $$
  f(x, y) =
  \begin{cases}
    x/y, & [y \smid x], \\
    \und, & \oth
  \end{cases}
  $$
  is partial recursive.
\end{lemma}

\begin{proof}
  Indeed,
  $$ f(x, y) = \min\b{ q \mid qy = x }. $$
\end{proof}

\section{Enumerability}

\begin{definition}
  A set $S \ss \N$ is \emph{enumerable}, iff there exists a TM that outputs all the elements of $S$ and only them, separated by commas.
\end{definition}

\begin{example}
  $\naught$ and $\N$ are enumerable.
  In general, all decidable sets are enumerable.
\end{example}

It is not long before until we give an example of an enumerable, non-decidable set.

\begin{definition}
  Let $S \ss \N$.
  Its \emph{semicharacteristic} function is the partial function
  $$ \daleth_S(x) := \begin{cases}
    1, & x \in S \\
    \und, & \oth.
  \end{cases}
  $$
\end{definition}

\begin{lemma}
  Let $S \ss \N$ be nonempty.
  The following are equivalent:
  \begin{enumerate}
    \item $S$ is enumerable.
    \item Its semicharacteristic function is computable.
    \item There exists a computable $f \col S \to \N$.
    \item There exists an initial segment $R \ss \N$ and a computable bijective $g \col R \to S$.
    \item There exists a computable surjective $h \col \N \to S$.
  \end{enumerate}
\end{lemma}

\begin{proof}[$1 \Rightarrow 2$]
  Run the TM until it outputs the element.
  We are comfortable with the prospect of it never doing that.
\end{proof}

\begin{proof}[$2 \Rightarrow 3$]
  Obvious.
\end{proof}

\begin{proof}[$3 \Rightarrow 4$]
  Via minimisation: let $g(0)$ be the smallest member of $S$, and let $g(x)$ for $x > 0$ be the smallest member of $S$ that is greater than $g(x-1)$. This function is not defined everywhere in case of a finite $S$.
\end{proof}

\begin{proof}[$4 \Rightarrow 5$]
  To deal with the case of a finite $S$, we put $h(x) := g(x)$ where $g$ is defined, and $h(x) = \max S$, if $x \ge \a{R}$.
\end{proof}

\begin{proof}[$5 \Rightarrow 1$]
  Build a TM that outputs $f(0), f(1), \dots$ 
\end{proof}

\begin{lemma}[Post's criterion]
  \label{Post's criterion}
    Let $S \ss \N$.
    $S$ is decidable iff both $S$ and $\ol S$ are enumerable.
\end{lemma}

\begin{proof}[$\Rightarrow$]
  Iterate over naturals and check inclusion for each.
  Output accordingly.
\end{proof}

\begin{proof}[$\Leftarrow$]
  To compute the characteristic function on $x \in \N$, we run machines for $S$ and $\ol S$ in parallel.
  The first to output $x$ corresponds to the correct answer.
\end{proof}

\subsection{Projections}

\begin{definition}
  Let $S \ss \N^n$. We call the set
  $$ \Proj_i S = \b{ x_i \mid \ex x_-, x_+ \col (x_-, x_i, x_+) \in S } $$
  a \emph{projection} of $S$ onto the coordinate $i$.
\end{definition}

\begin{lemma}[on projections]
  Let $P \ss \N$.
  The following are equivalent:
  \begin{enumerate}
    \item $P$ is enumerable.
    \item There exists a decidable $Q \ss \N^2$ such that $P$ is a projection of $Q$.
  \end{enumerate}  
\end{lemma}

\begin{proof}[$1 \Rightarrow 2$]
  Define $Q$ to be the set of pairs $(n, t)$ such that the number $n$ appears in the output of the enumerating machine of $P$ within $t$ steps.
\end{proof}

\begin{proof}[$2 \Rightarrow 1$]
  Using a Cantor enumeration, iterate over all members of $Q$, outputting their projections (onto the known coordinate).
\end{proof}

\begin{definition}
  A set $S \ss \N^n$ is \emph{enumerable}, iff its image under a Cantor enumeration of $\N^n$ is enumerable.
\end{definition}

\begin{lemma}[on graphs]
  Let $f \col \N \to \N$.
  The following are equivalent:
  \begin{enumerate}
    \item $f$ is computable.
    \item Its graph is enumerable.
  \end{enumerate} 
\end{lemma}

\begin{proof}[$1 \Rightarrow 2$]
  Since $f$ is defined everywhere, we may iterate over the naturals and output for each $x \in \N$ the pair $(x, f(x))$ (in the guise of its image under a Cantor enumeration).
\end{proof}

\begin{proof}[$2 \Rightarrow 1$]
  To compute $f(x)$, run the Turing machine for the graph until arriving at $(x, f(x))$.
  This will happen.
\end{proof}

\subsection{Universal functions}

\begin{definition}
  Let $D \ss \N^{m+1}$, and let $U \col D \to \N^n$ be a computable function.
  The function $U_k \col \Proj_{1, \dots, m} D \to \N^n$, defined, for $k \in \Proj_0 d$ and $x \in \N^m$ as
  $$ U_{k}(x) := U(k, x), $$
  we will call a \emph{section} of $U$.
  $U$ itself is said to be \emph{universal} for the class of functions
  $$ C = \b{ U_k \mid k \in \Proj_0 D }. $$
\end{definition}

\begin{lemma}
  There exists a universal function for the class $\Comp_m$.
\end{lemma}

\begin{proof}
  Let $U_k(x)$ be the output of the TM number $k$ on $x$. 
\end{proof}

\begin{lemma}
  There does not exist an everywhere defined universal function $U \col \N^2 \to \N$ for the class $\Comp_1^*$.
\end{lemma}

\begin{proof}
  By diagonal argument.
  Consider the function $n \mapsto U(n, n)+1$.
  It is computable, but differs from any section $U_k$ at $k$:
  $$ U_k(k) + 1 \ne U_k(k). $$
  A contradiction.
\end{proof}

\begin{remark}
  This proof fails for $\Comp_1$, since $U(k, k)$ may not exist.
\end{remark}

\begin{lemma}
  There exists a computable $f \col \N \to \N$ such that there does not exist an $F \col \N \to \N$ with
  $ F|_{\dom f} = f$.
\end{lemma}

\begin{proof}
  Take $f \col n \mapsto U(n,n)+1$, where $U$ is an (existent) universal function for $\Comp_1$.
  Suppose $F$ exists.
  \begin{itemize}
    \item If $n \in \dom f$, then $F(n) \ne U(n, n)$.
    \item If $n \not\in \dom f$, then $n \not\in \dom U$ and $n \in \dom F$.
  \end{itemize}
  Therefore, $F \ne U_n$ for any $n \in \N$.
  But $F \in \Comp_1^* \ss \Comp_1$, and $U$ is universal for $\Comp_1$.
  A contradiction.
\end{proof}

\subsection{An enumerable non-decidable set}

\begin{theorem}
  There exists an enumerable non-decidable set.  
\end{theorem}

\begin{proof}
  Let $f \col n \mapsto U(n,n)+1$ be as in the previous lemma.
  $S := \dom f$ is enumerable, as it is the domain of a computable function.
  We assert that $S$ is undecidable.
  Suppose otherwise.
  Put
  $$
  F(x) = \q{x \in S} f(x).
  $$
  Since the characteristic function $\q{x \in S}$ is computable, $F$ is computable and defined everywhere.
  This contradicts the previous lemma.
\end{proof}

\subsubsection{Some remarks on this proof}

\begin{itemize}
  \item In fact, the set
  $$ S = \b{ n \mid U(n, n) \text{ is defined}} $$
  is a guise of the classic diagonal argument on machines that do not accept themselves.
  \item $\ol S$ is not enumerable. If both $S$ and $\ol S$ were enumerable, they would be decidable (obvious, but we have met that on page \pageref{Post's criterion}).
  \item The domain of this $U \col {\ss \N^2} \to \N$ is an enumerable, undecidable set itself. It is enumerable, as it is a domain of a computable function, but the diagonal function $n \mapsto U(n, n)$ serves as a counterexample to decidability.
\end{itemize}

\begin{lemma}
  There exists a computable $f \col {\ss \N} \to 2$ that does not a have an everywhere defined computable continuation $F \col \N \to \N$ (so that $F|_{\dom f} = f$). 
\end{lemma}

\begin{proof}
  Put
  $$
  f(x) = \begin{cases}
    \q{U(x,x) = 0}, & \text{if $U(x, x)$ is defined,}\\
    \und & \oth.
  \end{cases}
  $$
  $F$ differs from any section of $U$ in the diagonal.
\end{proof}

\subsection{Unseparable enumerable sets}

\begin{lemma}
  There exist enumerable disjoint $X$ and $Y$ such that, if $Z$ is decidable and $Z \sps X$, then
  $$ Y \cap Z \ne 0. $$ 
\end{lemma}

Thus, $X$ and $Y$ cannot be `separated' by decidable sets.

\begin{proof}
  Let $f$ be as in the previous lemma, and put $X = f^{-1}(0)$, $Y = f^{-1}(1)$.
  Suppose there exists a decidable $Z$ such that $Z \sps X$ and $Z \cap Y = \naught$. 
  Now let
  $$
  F(x)
  = \begin{cases}
    f(x), & x \in Z, \\
    0 & x \not\in Z.
  \end{cases}
  $$
  Since $Z$ contains all the $x$s such that $f(x)$ is nonzero, $F$ continues $f$.
  But it is computable, in contradiction with the conclusion.
\end{proof}